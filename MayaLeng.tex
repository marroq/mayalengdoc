\documentclass[a4paper,openright,12pt]{article}
%%[11pt,twocolumn]
\title{\LARGE MayaLeng{}}
\author{\normalsize Douglas Figueroa \\ Alexander Baquiax}
\usepackage[spanish]{babel}
\date{} 
\usepackage[spanish]{babel}
\usepackage[utf8]{inputenc}

\begin{document}
\maketitle

\section{\'Indice}
\subsection{\'Indice de Figuras}
\subsection{\'Indice de Tablas}

\newpage

\section{Introducci\'on}
\newpage

\section{Objetivos}
\subsection{Objetivo General}
Crear una herramienta de traducci\'on de lenguas mayas.
\subsection{Objetivos Espec\'ificos}
\begin{itemize}
    \item Crear una aplicaci\'on m\'ovil (Android y iOS) para traducir lenguas mayas.
    \item Dar a conocer la cultura de nuestro pa\'is Guatemala.
    \item Brindar la capacidad a profesores y personas en general de traducir documentos para facilitar su trabajo en los lugares d\'onde imparten clases dentro de nuestro pa\'is y los cuales no dominan el espanol.
\end{itemize}
\newpage

\section{El Problema}
\subsection{¿Qu\'e es el problema?}
El problema analizado para querer desarrollar esta herramienta fue desde un punto de vista cultural, ya que en Guatemala existen muchas lenguas mayas, de las cuales pocas personas en el pa\'is dominan como m\'inimo una. Estas forman parte de nuestro patrimonio cultural. Alrededor del mundo Guatemala es conocida por los puntos tur\'isticos con los que contamos, regularmente estos se encuentran en los departamentos donde al menos existe una lengua Maya.  Como guatemaltecos deber\'iamos de poder comunicarnos con la gente de nuestro pa\'is, ya que en algunos departamentos no se habla mucho en espa\~nol sino que hablan en una lengua maya, sin embargo ese no es el caso. No hemos encontrado tantas herramientas claras que nos faciliten dicha tarea, queremos ser de los primeros en brindar dicha herramienta.\\

En este momento sugerimos kaqchikel, la forma en que solucionamos el problema de la comunicación es teniendo una aplicación m\'ovil la cu\'al har\'a una traducción en nuestro idioma natal a kaqchikel, sencillamente quien utilice la aplicación deberá ingresar el texto y en una casilla aparte aparecer\'a la traducci\'on del texto, algo similar a google translate.\\

Le ahorraremos al usuario tener que escribir algunas de las frases o palabras m\'as utilizadas, ya que tendremos una secci\'on donde diremos como se dicen esas frases, cosas b\'asicas como saludar, despedirse, hasta algo m\'as formal, un ejemplo ser\'ia preguntar d\'onde se encuentra el bano, preguntar el nombre de una persona y frases similares.\\
\newpage

\section{Estudio de Factibilidad}
\subsection{Factibilidad Funcional}
\subsection{Factibilidad T\'ecnica}
Pretendemos aprovechar el apogeo de los m\'oviles y tratar de crear una buena oportunidad para introducir nuestro proyecto.\\

Actualmente existen dos grandes sistemas operativos que domina la industria de los m\'oviles:. iOS y Android. Estamos conscientes de que desarrollar de forma nativa para ambas plataformas nos representar\'ia un poco m\'as de tiempo, que implicar\'ia restarle tiempo a la parte que en verdad es importante. Por lo cual usaremos IONIC, framework que nos permite desarrollar de manera sencilla aplicaciones m\'oviles usando las tendencias de Responsive Web Design. \\

Lo interesante de este proyecto no radica en la aplicaci\'on m\'ovil, de hecho la aplicaci\'on s\'olo ser\'a una forma de consumir nuestro verdadero sistema.\\

La idea de este proyecto radica en hacer un compilador que pueda usar una gram\'atica y una fuente de palabras de esa gram\'atica, y traducir de ellas las palabras al español. En pocas palabras nuestro proyecto radica en hacer un compilador de idiomas mayas.\\

Durante el transcurso de nuestra carrera ya hicimos un compilador, con todas las fases b\'asicas que uno de ellos debe tener. Ahora nos pusimos el reto de hacer un compilador gen\'erico. Para esta primera versi\'on usaremos dos idiomas Mayas.\\

Creemos tener los conocimientos necesarios para la construcci\'on de esto. Uno de los inconvenientes m\'as grandes era que ninguno de los dos sabemos hablar un idioma Maya, pero nos apoyamos en la documentaci\'on que distintas personas a lo largo de la historia han construido.\\ 

La idea es construir un API al que se le pueda dar como input un texto, p\'arrafo, documento y que la salida sea otro documento pero con texto traducido. Ahora tenemos una base de datos con 15 mil palabras aproximadamente del Kaqchikel. La DB esta montada en un DBMS MySQL. El API fue construido en Symfony2.\\

El core de traducci\'on ser\'a hecho en Java,  usando herramientas de parseo y an\'alisis sint\'actico como Flex.

\subsection{Factibilidad Econ\'omica}
\newpage

\section{Anexo}
\newpage

\section{Glosario}
\newpage

\end{document}